\documentclass{article}
\usepackage{enumerate}
\usepackage[letterpaper, margin=1.5in]{geometry}

\newcommand{\namesigdate}[2][5cm]{%
  \begin{tabular}{@{}p{#1}@{}}
    #2 \\[2\normalbaselineskip] \hrule \\[0pt]
    {\small \textit{Date}} \\[2\normalbaselineskip] \hrule \\[0pt]
    {\small \textit{Signature}}
  \end{tabular}
}

\renewcommand\thesection{\Alph{section}}
\renewcommand\thesubsection{\thesection.\arabic{subsection}}
\renewcommand\theparagraph{\thesection.\thesubsection.\thesubsubsection.(\roman{paragraph})}

\begin{document}

\title{Operating Agreement of Padberry LLC, a Member-Managed Limited Liability Company}
\date{Effective January 17, 2018}
\maketitle

	% Section A - Preliminary Provisions
	\section{Preliminary Provisions}

			\subsection{Effective Date}
			This Operating Agreement of Padberry LLC, effective on the the seventeenth of January of 2018, is adopted by the members whose signatures appear at the end of this agreement.

			\subsection{Formation}
			This limited liability company (LLC) was formed by filing Articles of Organization, a Certificate of Formation, or a similar organizational document with the state of Colorado's LLC filing office on January 17, 2018. Unless a delayed effective date was specified when the Articles, Certificate of Formation, or similar document was filed, the legal existence of this LLC commenced on the date of such filing. A copy of this organizational document has been placed in the LLC's records book.

			\subsection{Name}
			The formal name of this LLC is as stated above. However, this LLC may do business under a different name by complying with the state's fictitious or assumed business name statutes and procedures.

			\subsection{Registered Office and Registered Agent}
			The registered office address of this LLC is:

			\begin{center}
				\begin{tabular}{l}
					2065 North Downing Street \#7\\
					Denver, Colorado 80205\\
				\end{tabular}
			\end{center}
			\noindent The registered agent of this LLC is:
			\begin{center}
				\begin{tabular}{l}
					Rebecca Wood\\
				\end{tabular}
			\end{center}

			The registered agent and/or office of this LLC may be changed from time to time as the members may see fit, by filing a change of registered agent or office statement with the state LLC filing office. It will not be necessary to amend this provision of the Operating Agreement if and when such changes are made.

			\subsection{Business Purpose}
			The specific business purposes and activities contemplated by the founders of this LLC at the time of initial signing of this agreement consist of the following: Create, market, and sell products suitable for children including but not limited to clothing, accessories, and literature.
			
			\indent It is understood that the foregoing statement of powers shall not serve as a limitation on the powers or abilities of this LLC, which shall be permitted to engage in any and all lawful business activities. If this LLC intends to engage in business activities outside the state of its formation that require the qualification of the LLC in other states, it shall obtain such qualification before engaging in such out-of-state activities.

			\subsection{Duration of LLC}
			The duration of this LLC shall be perpetual. This LLC shall terminate when a proposal to dissolve the LLC is adopted by the membership of this LLC or when this LLC is otherwise terminated in accordance with law.
		

	% Section B - Membership Provisions
	\section{Membership Provisions}

			\subsection{Non-liability of Members}
			No member of this LLC shall be personally liable for the expenses, debts, obligations, or liabilities of the LLC, or for claims made against it.

			\subsection{Reimbursement of Organizational Costs}
			Members shall be reimbursed by the LLC for organizational expenses paid by the members. The LLC shall be authorized to elect to deduct organizational expenses and start-up expenditures and amortize as permitted by the Internal Revenue Code and as may be advised by the LLC's tax adviser.

			\subsection{Management}
			This LLC shall be managed exclusively by all of its members.

			\subsection{Members' Capital Interests}
			A member's capital interest in this LLC shall be computed as a fraction, the numerator of which is the total of a member's capital account and the denominator of which is the total of all capital accounts of all members.

			\subsection{Membership Voting}
			Except as otherwise may be required by the Articles of Organization, Certificate of Formation, or a similar organizational document; other provisions of this Operating Agreement; or under the laws of this state; each member shall vote on any matter submitted to the membership for approval by the managers of this LLC in proportion to the member's capital interest in this LLC. Further, unless otherwise stated in another provision of this Operating Agreement, the phrase ``majority of members'' means a majority of members whose combined capital interests in this LLC represent more than 50\% of the capital interests of all members in this LLC, and a majority of members, so defined, may approve any item of business brought before the membership for a vote.

			\subsection{Compensation}
			Members shall not be paid as members of the LLC for performing any duties associated with such membership, including management of the LLC. Members may be paid, however, for any services rendered in any other capacity of the LLC, whether as officers, employees, independent contractors, or otherwise.

			\subsection{Members' Meetings}
			The LLC shall not provide for regular members' meetings. However, any member may call a meeting by communicating his or her wish to schedule a meeting to all other members. Such notification may be in person or in writing, or by telephone, or other form of electronic communications reasonably expected to be received by a member, and the other members shall then agree, either personally, in writing, or by telephone, or other form of electronic communication to the member calling the meeting, to meet at a mutually acceptable time and place. Notice of the business to be transacted at the meeting need not be given to members by the member calling the meeting, and any business may be discussed and conducted at the meeting.

			\index If all members cannot attend a meeting, it shall be postponed to a date and time when all members can attend, unless all members who do not attend have agreed in writing to the holding of the meeting without them. If a meeting is postponed, and the postponed meeting cannot be held either because all members do not attend the postponed meeting or the non-attending members have not signed a written consent to allow the postponed meeting to be held without them, a second postponed meeting may be held at a date and time announced at the first postponed meeting. The data and time of the second postponed meeting shall also be communicated to any members not attending the first postponed meeting. The second postponed meeting may be held without the attendance of all members as long as a majority of the capital interests of the membership of this LLC is in attendance at the second postponed meeting. Written notice of the decisions or approvals made at this second postponed meeting shall be mailed or delivered to each non-attending member promptly after the holding of the second postponed meeting.

			\index Written minutes of the discussions and proposals presented at a members' meeting, and the votes taken and matter approved at such meeting, shall be taken by one of the members or a person designated at the meeting. A copy of the minutes of the meeting shall be placed in the LLC's records book after the meeting.

			\subsection{Membership Certificates}
			This LLC shall be authorized to obtain and issue certificates representing or certifying membership interests in this LLC. Each certificate shall show the name of the LLC and the name of the member, and shall state that the person named is a member of the LLC and is entitled to all the rights granted members of the LLC under the Article of Organization, Certificate of Formation, or a similar organizational document; this Operating Agreement; and provisions of law. Each membership certificate shall be consecutively numbered and signed by each of the current members of this LLC. The certificates shall include any additional information considered appropriate for inclusion by the members on membership certificates.

			\index In addition to the above information, all membership certificates shall bear a prominent legend on their face or reverse side stating or summarizing any transfer restrictions that apply to memberships in this LLC under the Articles of Organization, Certificate of Formation, or a similar organizational document, and/or this Operating Agreement, and the address where a member may obtain a copy of these restrictions upon request from this LLC.

			\index The records book of this LLC shall contain a list of the names and addresses of all persons to whom certificates have been issued, show the date of issuance of each certificate, and record the date of all cancellations or transfers of membership certificates by members or the LLC.

			\subsection{Other Business By Members}
			Each member shall agree not to own an interest in, manage, or work for another business, enterprise, or endeavor, if such ownership or activities would compete with this LLC's business goals, mission, profitability, or productivity, or would diminish or impair the member's ability to provide maximum effort and performance in accomplishing the business objectives and, if applicable, managing the business of this LLC.

			\subsection{Admission of New Members}
			Except as otherwise provided in this agreement, a person or entity shall not be admitted into membership in this LLC unless each member consents in writing to the admission of the new member. The admission of new members into this LLC who have been transferred, or wish to be transferred, a membership interest in this LLC by an existing member of this LLC is covered by separate provisions in this Operating Agreement.
		

	% Section C - Tax and Financial Provision
	\section{Tax and Financial Provisions}

		\subsection{Tax Classification of LLC}
		The members of this LLC intend that this LLC be initially classified as a partnership for federal and, if applicable, state income tax purposes. It is understood that all members may agree to change the tax treatment of this LLC by signing, or authorizing the signature of, IRS Form 8832, \emph{Entity Classification Election}, and filing it with the IRS and, if applicable, the state tax department within the prescribed time limits.

		\subsection{Tax Year and Accounting Method}
		The tax year of this LLC shall end on the last day of the month of December. The LLC shall use the cash method of accounting. Both the tax year and the accounting period of the LLC may be changed with the consent of all members if the LLC qualifies for such change, and may be effected by the filing of appropriate forms with the IRS and state tax offices.

		\subsection{Tax Matters Partner}
		If this LLC is required under Internal Revenue Code provisions or regulations, it shall designate from among its members a “tax matters partner” in accordance with the Internal Revenue Code Section 6231(a)(7) and corresponding regulations, who will fulfill this role by being the spokesperson for the LLC in dealings with the IRS as required under the Internal Revenue Code and Regulations, and who will report to the members on the progress and outcome of these dealings.

		\subsection{Annual Income Tax Returns and Reports}
		Within 60 days after the end of each tax year of the LLC, a copy of the LLC's state and federal income tax returns for the preceding tax year shall be mailed or otherwise provided to each member of the LLC, together with any additional information and forms necessary for each member to complete his or her individual state and federal income tax returns. This additional information shall include a federal (and, if applicable, state) Schedule K-1 (Form 1065 - \emph{Partner's Share of Income, Credits, Deductions}) or equivalent income tax reporting form, as well as a financial report, which shall include a balance sheet and profit and loss statement for the prior tax year of the LLC.

		\subsection{Bank Accounts}
		The LLC shall designate one or more banks or other institutions for the deposit of the funds of the LLC, and shall establish savings, checking, investment, and other such accounts as are reasonable and necessary for its business and investments. One or more members of the LLC shall be designated with the consent of all members to deposit and withdraw funds of the LLC, and to direct the investment of funds from, into, and among such accounts. The funds of the LLC, however and wherever deposited or invested, shall not be commingled with the personal funds of any members of the LLC.

		\subsection{Title to Assets}
		 All personal and real property of this LLC shall be held in the name of the LLC, not in the names of individual members.
	

	% Section D - Capital Provisions
	\section{Capital Provisions}

			\subsection{Capital Contributions by Members}
			Members shall make the following contributions of cash, property, or services to the LLC, on or by specified dates, as shown next to each member's name below. The fair market values of items of property or services as agreed between the LLC and the contributing members are also shown below:
				\begin{center}
					\begin{tabular}{l c c}
						Name of Member: & Jonathan Bowen & Rebecca Wood\\
						Description of Payment: & Cash & Cash\\
						Value of Capital Payment: & \$200 & \$200\\
						Date of Payment: & February 1, 2018 & February 1, 2018\\
					\end{tabular}
				\end{center}
			\subsection{Additional Contributions by Members}
			The members may agree, from time to time by unanimous vote, to require the payment of additional capital contributions by the members, on or by a mutually agreeable date.

			\subsection{Failure to Make Contributions}
			If a member fails to make a required capital contribution with the time agreed for a member's contribution, the remaining members may, by unanimous vote, agree to reschedule the time for payment of the capital contribution by the late-paying member, setting any additional repayment terms, such as a late payment penalty, rate of interest to be applied to the unpaid balance, or other monetary amount to be paid by the delinquent member, as the remaining members decide. Alternatively, the remaining members may, by unanimous vote, agree to cancel the membership of the delinquent member, provided any prior partial payments of capital made by the delinquent member are refunded promptly by the LLC to the member after the decision is made to terminate the membership of the delinquent member.

			\subsection{No Interest on Capital Contributions}
			No interest shall be paid on funds or property contributed as capital to this LLC, or on funds reflected in the capital accounts of the members.

			\subsection{Capital Account Bookkeeping}
			A capital account shall be set up and maintained on the books of the LLC for each member. It shall reflect each member's capital contribution to the LLC, increased by each member's share of profits in the LLC, decreased by each member's share of losses and expenses of the LLC, and adjusted as required in accordance with applicable provisions of the Internal Revenue Code and corresponding income tax regulations.

			\subsection{Consent to Capital Contribution Withdrawals and Distributions}
			Members shall not be allowed to withdraw any part of their capital contributions or to receive distributions, whether in property or cash, except as otherwise allowed by this agreement and, in any case, only if such withdrawal is made with the written consent of all members.

			\subsection{Allocations of Profits and Losses}
			Except as otherwise provided in the Articles of Organization, Certificate of Formation, or a similar organizational document, or this Operating Agreement, no member shall be given priority or preference with respect to other members in obtaining a return of capital contributions, distributions, or allocations of the income, gains, losses, deductions, credits, or other items of the LLC. Except as otherwise provided in the Articles of Organization, Certificate of Formation, a similar organizational document, or this Operating Agreement, the profits and losses of the LLC, and all items of its income, gain, loss, deduction, and credit shall be allocated to members according to each member's capital interest in this LLC.

			\subsection{Allocation of Distribution of Cash to Members}
			Cash from LLC business operations, as well as cash from a sale or other disposition of LLC capital assets, may be allocated and distributed from time to time to members in accordance with each member's capital interest in the LLC, as may be decided by a majority of the capital interests of the members.

			\subsection{Allocation of Noncash Distributions}
			If proceeds consist of property other than cash, the members shall decide the value of the property and allocate such value among the members in accordance with each member's capital interest in the LLC. If such noncash proceeds are later reduced to cash, such cash may be distributed among the members according to the distribution of cash allocations provisions in this agreement.

			\subsection{Allocation and Distribution of Liquidation Proceeds}
			Regardless of any other provision in this agreement, if there is a distribution or liquidation of this LLC, or when any member's interest is liquidated, all items of income and loss shall be allocated to the members' capital accounts, and all appropriate credits and deductions shall then be made to these capital accounts before any final distribution is made. A final distribution shall be made to members only to the extent of, and in proportion to, any positive balance in each member's capital account.

		% Section E - Membership Withdrawal and Transfer Provisions
		\section{Membership Withdrawal and Transfer Provisions}

			\subsection{Withdrawal of Members}
			A member may withdraw from this LLC by giving written notice to all other members at least 90 days before the date the withdrawal is to be effective. In the event of such withdrawal, the LLC shall pay the departing member the fair value of his or her LLC interest, less any amounts owed by the member to the LLC. The departing and remaining members shall agree at the time of departure on the fair value of the departing member's interest and the schedule of payments to be made by the LLC to the departing member, who shall receive payment for his or her interest within a reasonable time after departure from the LLC. If the departing and remaining members cannot agree on the value of the departing member's interest, they shall select an appraiser, who shall determine the current value of the departing member's interest. This appraised amount shall be the fair value of the departing member's interest, and shall form the basis of the amount to be paid to the departing member.

			\subsection{Restrictions on the Transfer of Membership}
			Notwithstanding any other provision of this agreement, a member shall not transfer his or her membership in the LLC unless all of the nontransferring LLC members first agree in writing to approve the admission of the transferee into this LLC. Further, no member may encumber a part or all of his or her membership in the LLC by mortgage, pledge, granting of a security interest, lien, or otherwise, unless the encumbrance has first been approved in writing by all other members of the LLC.
			\indent Notwithstanding the above provision, any member shall be allowed to assign an economic interest in his or her membership to another person without approval of the other members. Such an assignment shall not include a transfer of the member's voting or management rights in this LLC, and the assignee shall not become a member of the LLC.

		% Section F - Dissolution Provisions
		\section{Dissolution Provisions}

			\subsection{Events That Trigger Dissolution of the LLC}
			The following events shall trigger a dissolution of the LLC, except as provided:

				\subsubsection{Dissociation of a Member.}
				The dissociation of a member, which means the death, incapacity, bankruptcy, retirement, resignation, or expulsion of a member, or any other event that terminates the continued membership of a member, shall cause a dissolution of this LLC only if and as provided below:
					\begin{enumerate}[{\bf (i)}]
						\item{\bf If a vote must be taken under state law to avoide dissolution.}
						If, under provisions of state law, a vote of the remaining LLC members is required to continue the existence of this LLC after the dissociation of a member, the remaining members shall affirmatively vote to continue the existence of this LLC within the period, and by the number of votes of remaining members, that may be required under such provisions. If such a vote is required, but the period or number of votes requirement is not specified under such provisions, all remaining members must affirmatively vote to a continuation of this LLC within 90 days from the date of the date of dissociation of the member. If the affirmative vote of the remaining members is not obtained under this provision, this LLC shall dissolve under the appropriate procedures specified under state law.

						\item{\bf If a vote is not required under state law to avoid dissolution.}
						If provisions of state law do not require such a vote of remaining members to continue the existence and/or business of the LLC after the dissociation of a member, and/or allow the provisions of this Operating Agreement to take precedence over state law provisions relating to the continuance of the LLC following the dissociation of a member, this LLC shall continue its existence and business following such dissociation of a member without the necessity of taking a vote of the remaining members. Notwithstanding the above, if the LLC is left with fewer members than required under state law for the operation of an LLC following the dissociation of a member of this LLC, the LLC shall elect or appoint a member in accordance with any provisions of state law regarding such election or appointment. If such election or appointment is not made within the time period specified under state law, or, if no time period is specified under state law and the LLC makes no election or appointment within 90 days following the date of dissociation of the member, this LLC shall dissolve under the appropriate procedures specified under state law.					
					\end{enumerate}

				\subsubsection{Expiration of LLC Term.}
				The expiration of the term of existence of the LLC if such term is specified in the Articles of Organization, Certificate of Formation, a similar organizational document, or this Operating Agreement, shall cause the dissolution of this LLC.
				
				\subsubsection{Written Agreement to Dissolve.}
				The written agreement of all members to dissolve the LLC shall cause a dissolution of this LLC.

				\subsubsection{Entry of Decree.}
				The entry of decree of dissolution of the LLC under state law shall cause a dissolution of this LLC.

			If the LLC is to dissolve according to any of the above provisions, the members and, if applicable, managers, shall wind up the affairs of the LLC, and take other actions appropriate to complete a dissolution of the LLC in accordance with applicable provisions of state law.

		% Section G - General Provisions
		\section{General Provisions}

			\subsection{Officers}
			The LLC may designate one or more officers, such as a President, Vice President, Secretary, and Treasurer. Persons who fill these positions need not be members of the LLC. Such positions may be compensated or noncompensated according to the nature and extent of the services rendered for the LLC as a part of the duties of each office. Ministerial services only as a part of any officer position will normally not be compensated, such as the performance of officer duties specified in this agreement, but any officer may be reimbursed by the LLC for out-of-pocket expenses paid by the officer in carrying out the duties of his or her office.

			\subsection{Records}
			The LLC shall keep at its principal business address a copy of all proceedings of membership meetings, as well as books of account of the LLC's financial transactions. A list of the names and addresses of the current membership of the LLC also shall be maintained at this address, with notations on any transfers of members' interest to nonmembers or persons being admitted into membership in the LLC.

			\indent Copies of the LLC's Articles of Organization, Certificate of Formation, or a similar organizational document; a signed copy of this Operating Agreement; and the LLC's tax returns for the preceding three tax years shall be kept at this address containing any of the following information that is applicable to this LLC:

				\begin{itemize}
					\item the amount of cash or a description and value of property contributed or agreed to be contributed as capital to the LLC by each member

					\item a schedule showing when any additional capital contributions are to be made by members to this LLC

					\item a statement or schedule, if appropriate, showing the rights of members to receive distributions representing a return of part or all of members' capital contributions, and

					\item a description of events, or the date, when the legal existence of the LLC will terminate under provisions in the LLC's Articles of Organization, Certificate of Formation, or a similar organizational document, or this Operating Agreement.

				\end{itemize}

			If one or more of the above items is included or listed in this Operating Agreement, it will be sufficient to keep a copy of this agreement at the principal business address of the LLC without having to prepare and keep a separate record of such item or items at this address.
			\indent Any member may inspect any and all records maintained by the LLC upon reasonable notice to the LLC. Copying of the LLC's records by members is allowed, but copying costs shall be paid for by the requesting member.


			\subsection{All Necessary Acts}
			The members and officers of this LLC are authorized to perform all acts necessary to perfect the organization of this LLC and to carry out its business operations expeditiously and efficiently. The Secretary of the LLC, or other officers, or all members of the LLC, may certify to other businesses, financial institutions, and individuals as to the authority of one or more members or officers of this LLC to transact specific items of business on behalf of the LLC.

			\subsection{Mediation and Arbitration of Disputes Among Members}
			In any dispute over the provisions of this Operating Agreement and in other disputes among members, if the members cannot resolve the dispute to their mutual satisfaction, the matter shall be submitted to mediation. The terms and procedure for mediation shall be arranged by the parties to the dispute.

			\indent If good-faith mediation of a dispute proves impossible or if an agreed-upon mediation outcome cannot be obtained by the members who are parties to the dispute, the dispute may be submitted to arbitration in accordance with the rules of the American Arbitration Association. Any party may commence arbitration of the dispute by sending a written request for arbitration to all other parties to the dispute. The request shall state the nature of the dispute to be resolved by arbitration, and, if all parties to the dispute agree to arbitration, arbitration shall be commenced as soon as practical after such parties receive a copy of the written request.

			\indent All parties shall initially share the cost of arbitration, but the prevailing party or parties may be awarded attorney fees, costs, and other expenses of arbitration. All arbitration decisions shall be final, binding, and conclusive on all the parties to arbitration, and legal judgment may be entered based upon such decision in accordance with applicable law in any court having jurisdiction to do so.

			\subsection{Entire Agreement}
			This Operating Agreement represents the entire agreement among the members of this LLC, and it shall not be amended, modified, or replaced except by a written instrument executed by all the parties to this agreement who are current members of this LLC as well as any and all additional parties who became members of this LLC after the adoption of this agreement. This agreement replaces and supersedes all prior written and oral agreements among any and all members of this LLC.

			\subsection{Severability}
			If any provision of this agreement is determined by a court or arbitrator to be invalid, unenforceable, or otherwise ineffective, that provision shall be severed from the rest of this agreement, and the remaining provisions shall remain in effect and enforceable.

		\vspace{1.5in}

		% Section H - Signature of Members and Spouses of Members
		\section{Signatures of Members and Spouses of Members}

			\subsection{Execution of Agreement}
			In witness whereof, the members of this LLC sign and adopt this agreement as the Operating Agreement of this LLC.\\
			\vspace*{.25 cm}

			\noindent \namesigdate{Jonathan Henry Bowen} \hfill \namesigdate{Rebecca Gail Wood}

			\vspace*{.25 cm}

%			\subsection{Consent of Spouses}
%			The undersigned, if any, are spouses of above-signed members of this LLC. These spouses have read this agreement and agree to be bound by its terms in any matter in which they have a financial interest, including restrictions on the transfer of memberships and the terms underwhich membership in this LLC may be sold or otherwise transferred.\\
%
%			\vspace*{.5 cm}
%			\noindent \namesigdate{None}





\end{document}